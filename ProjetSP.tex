\documentclass[12pt]{article}
%\usepackage{natbib}
\usepackage[french]{babel}
\usepackage{url}
\usepackage[utf8x]{inputenc}
\usepackage{graphicx}
\graphicspath{{images/}}
\usepackage{parskip}
\usepackage{fancyhdr}
\usepackage{vmargin}
\usepackage{xcolor}
\usepackage{bbm}
\usepackage{amsmath,amsthm,amssymb,latexsym,amsfonts}
\usepackage{dsfont}
\usepackage{stmaryrd}
\usepackage{systeme}
\usepackage{enumitem}
\usepackage{pifont}
%\usepackage[cache=false]{minted}
%\definecolor{LightGray}{gray}{0.95}
\usepackage{autobreak}
\usepackage{hyperref}

\title{Théorème fondamental des polynômes symétriques}
\author{PIARD A. - JACQUET R. - CARVAILLO T.}
\date{\today}

\makeatletter
\let\thetitle\@title
\let\theauthor\@author
\let\thedate\@date
\makeatother

\pagestyle{fancy}
\fancyhf{}
\rhead{\fontsize{7}{7} \selectfont \theauthor}
\lhead{\fontsize{10}{10} \selectfont \thetitle}
\cfoot{\thepage}
\def\dotfill#1{\cleaders\hbox to #1{.}\hfill}
\newcommand\dotline[2][.5em]{\leavevmode\hbox to #2{\dotfill{#1}\hfil}}

%définition commande présentation fonction
\newcommand{\fonction}[5]{
\begin{displaymath}
\begin{array}{l|rcl}
\displaystyle
#1 : & #2 & \longrightarrow & #3 \\
    & #4 & \longmapsto & #5
\end{array}
\end{displaymath}
}
%fin définition
\theoremstyle{remark}\newtheorem{note}{Note}
\theoremstyle{remark}\newtheorem{nota}{Notation}

\newcommand{\M}{\mathbbm{M}}
\newcommand{\N}{\mathbbm{N}}
\newcommand{\Z}{\mathbbm{Z}}
\newcommand{\Q}{\mathbbm{Q}}
\newcommand{\R}{\mathbbm{R}}
\newcommand{\C}{\mathbbm{C}}
\newcommand{\G}{\mathbbm{G}}
\newcommand{\K}{\mathbbm{K}}
\newcommand{\F}{\mathbbm{F}}
\newcommand{\Fq}{\mathbbm{F}_q}
\newcommand{\Fqn}{\mathbbm{F}_{q^n}}
\newcommand{\Fp}{\mathbbm{F}_p}
\newcommand{\ord}{\preccurlyeq}

%fin définition


% de jolies accolades
\newcommand{\accolade}[2]{
\begin{displaymath}
%#1 = \left\{
    \begin{array}{ll}
       #1 %& \mbox{si }
       #2 %& \mbox{sinon.}
    \end{array}
\right.
\end{displaymath}
}
% de jolies accolades


\newtheorem{theorem}{Théorème}
\newtheorem{corollaire}{Corollaire}
\newtheorem{lemma}{Lemme}
\newtheorem{prop}{Proposition}
\theoremstyle{definition}
\newtheorem{definition}{Définition}
\newtheorem{example}{Exemple}
\newtheorem*{examples}{Exemples}
\newtheorem{exo}{Exercice}	
\newtheorem{coro}{Corollaire}	
\newtheorem{rem}{Remarque}
\newtheorem{crit}{Critère}
\newtheorem{bg}{A l'attention des bg, question}

\begin{document}
%%%%%%%%%%%%%%%%%%%%%%%%%%%%%%%%%%%%%%%%%%%%%%%%%%%%%%%%%%%%%%%%%%%%%%%%%%%%%%%%%%%%%%%%%

\begin{titlepage}
	\centering
    \vspace*{0.5 cm}
    \textsc{\LARGE Projet de Systèmes Polynomiaux.\\
    \vspace{12pt}
2020-2021}\\[1.0 cm]
    \dotline[15pt]{15cm}\\
	%\includegraphics[scale = 2.2]{logo.png}
	\dotline[15pt]{15cm}\\
	\vspace{1.5cm}
	\textsc{\Large Faculté des Sciences et Techniques}\\
	\textsc{\large Master 1 - Maths. CRYPTIS}\\[1.0 cm]
	\rule{\linewidth}{0.2 mm} \\[0.4 cm]
	{ \huge \bfseries \color{blue} \thetitle}\\
	\rule{\linewidth}{0.2 mm} \\[1.5 cm]
	
	\begin{minipage}{0.4\textwidth}
		\begin{flushleft} \large
			\emph{A l'attention de :}\\
			M. LICKTEIG\\
			\phantom{a}\\
			\phantom{a}\\
		\end{flushleft}
	\end{minipage}
	\begin{minipage}{0.5\textwidth}
    	\begin{flushright} \large
		\emph{Rédigé par :}\\
		PIARD A.\\
		JACQUET R.\\
		CARVAILLO T.\\
		\end{flushright}
	\end{minipage}\\[2 cm]
\end{titlepage}

%%%%%%%%%%%%%%%%%%%%%%%%%%%%%%%%%%%%%%%%%%%%%%%%%%%%%%%%%%%%%%%%%%%%%%%%%%%%%%%%%%%%%%%%%

\tableofcontents
\pagebreak


\section{Rappels sur les Corps Finis} 
Soit $\K$ un corps quelconque et soit $\varphi$ le morphisme suivant :
\begin{center}
$
\begin{array}{l|rcl}
\varphi : & \Z & \longrightarrow & \K \\
    & n & \longmapsto & n\cdot 1_{\K}
\end{array}
$
\end{center}
\vspace{12pt}
\begin{definition}
Soit $\K$ un corps quelconque. Toute partie $\mathcal{P}$ de $\K$ vérifiant :
\begin{itemize}
\item $\mathcal{P}$ est non vide et est une partie stable pour $+$ et $\times$ de $\K$ et $\mathcal{P}$ muni des lois induites par celles de $\K$ est lui-même un corps.
\item $\mathcal{P}$ est un sous-anneau de $\K$, $1 \in \mathcal{P}$ et $(p \in \mathcal{P}^{*} = \mathcal{P} - \{0 \} \Rightarrow p^{-1} \in \mathcal{P}^{*})$.
\item $\mathcal{P}$ est un sous-groupe de $(\K, +)$ et $\mathcal{P}^{*}$ muni de la loi $\times$ est un sous-groupe multiplicatif $(\K^{*}, \times)$.
\end{itemize}
est appelée sous-corps de $\K$.
\end{definition}
\vspace{12pt}
\begin{definition}
Soit $\K$ un corps quelconque.
\begin{itemize}
\item $\K$ est dit premier s'il ne contient aucun sous-corps strict.
\item Si $\K$ est un corps, le sous-corps de $\K$ engendré par $1_{\mathbbm{K}}$ est un corps premier, c'est le sous-corps premier de $\K$.
\end{itemize}
\end{definition}
\vspace{12pt}
Le noyau du morphisme $\varphi$ est un idéal de $\Z$ et donc de la forme $k\Z$ pour $k \in \Z$. Par le premier théorème d'isomorphisme on a Im$(\varphi) \cong \Z / n \Z$. Par intégrité de $\Z / n \Z$, $n=0$ ou $n$ est un nombre premier. Si $n=0$ alors $\varphi$ est injective et donc le sous-corps premier de $\K$ est isomorphe à $\Q$. Si $n \neq 0$ alors le sous-corps premier est isomorphe à $\Z / n \Z$ et $n$ s'appelle la \textbf{caractéristique} de $\K$. 
%On désignera dorénavant par $\K$ un corps fini de caractéristique $p$ avec $p$ un nombre premier.
\\

\begin{definition}
Soient $L$ et $\K$ deux corps. Si $L/K$ est une extension de corps alors $L$ est un espace vectoriel sur $K$, où l'addition vectorielle est l'addition dans $L$ et la multiplication par un scalaire $K \times L$ est la restriction à $K \times L$ de la multiplication dans $L$. La dimension du $K$-espace vectoriel $L$ est appelée le degré de l'extension et est notée $[L:K]$.
\end{definition}
\vspace{12pt}
\begin{definition}
Soit $P$ un polynôme sur un corps $K$. On appelle corps de décomposition de $P$ sur $K$ une extension $L$ de $K$ telle que :
\begin{itemize}
\item dans $L[X]$, $P(X)$ est produit de facteurs de degré $1$,
\item les racines de $P(X)$ engendrent $L$.
\end{itemize}
\end{definition}
\vspace{12pt}
\begin{prop}
Soit $P$ un polynôme sur un corps $K$. Alors $P$ admet un corps de décomposition, unique à $K$-isomorphisme près.
\end{prop}
\vspace{30pt}
\begin{prop}\hspace{12pt}
\begin{itemize}
\item Le cardinal de $\K$ est une puissance de $p$.
\item Réciproquement, pour tout $n \in \N^{*}$, il existe un corps $\K$ de cardinal $p^n$. En outre $\K$ est unique à isomorphisme près.
%On appelle corps de décomposition de P, la plus petite extension de K contenant toutes les racines de P (item 1)
\end{itemize}
\end{prop}
%\vspace{12pt}
\begin{proof}\hspace{12pt}
\begin{itemize}
\item Puisque le sous-corps premier de $\K$ est isomorphe à $\Z / p \Z$, alors $\K$ est naturellement muni d'une structure de $\Z / p \Z$-espace vectoriel. On note $n = [ \K : \Z / p \Z ]$. Alors $\# \K = \# (\Z / p \Z)^n = p^n$.
\item Soit $n \in \N^{*}$. Si $\K$ est un corps fini de cardinal $p^n$, alors $\K$ est le corps de décomposition de $X^{p^n} - X$ sur $\Z / p \Z$ : en effet, puisque pour tout $x \in \K$, $x$ est racine de $X^{p^n} - X$ alors $X^{p^n} - X$ possède ses $p^n$ racines dans $\K$.\\
Réciproquement, soit $K$ le corps de décomposition de $X^{p^n}$ sur $\Z / p \Z$. Soit $\mathcal{K}$ l'ensemble des éléments de $K$ qui sont racines de $X^{p^n} - X$. On vérifie que $\mathcal{K}$ est un sous-corps de $K$. Puisque $1_K \in \mathcal{K}$, et si $x,y \in \mathcal{K}$ alors $x^{p^n}= x$ et $y^{p^n} = y$, donc $(x+y)^{p^n} x + y$ et $(xy^{-1})^{p^n} = xy^{-1}$, si bien que $x + y, xy^{-1} \in \mathcal{K}$. Par ailleurs la dérivée formelle, $(X^{p^n} - X)' = -1$ est premier avec $X^{p^n} - X$ donc les racines de $X^{p^n} - X$ sont simples. On en déduit alors que $\# \mathcal{K} = p^n$. Finalement $K = \mathcal{K}$ est un corps à $p^n$ éléments et il est unique à isomorphisme près en vertu de l'unicité du corps de décomposition de $X^{p^n} - X$ sur $\Z / p \Z$.
\end{itemize}
\end{proof}
On notera dorénavant $\F_q$ \textbf{le} corps fini à $q = p^n$ éléments.
\subsection{Construction}
Soit $P \in \F_p [X]$ un polynôme irréductible sur $\F_p$. On note $n = $ deg$(P)$. Puisque $P$ est irréductible, l'idéal $(P)$ est donc maximal%est-ce qu'on le montre ?
. Le quotient $\F_p [X] / (P)$ est le corps de rupture de $P$ sur $\F_p$ de cardinal $p^n$. Afin de montrer que l'on peut toujours construire les corps finis nous allons montrer que pour tout $n \in \N^{*}$ il existe un polynôme irréductible sur $\F_p$ de degré $n$.

\begin{prop}
Soit $n \in \N^{*}$, on définit $\mathcal{P} (n,p)$ par
\begin{center}
$\mathcal{P} (n,p) = \{ P \in \F_p [X]$, $P$ unitaire, irréductible de degré $n \}$.
\end{center}
Alors pour tout $n \in \N^{*}$ on a,
\begin{center}
$\displaystyle X^{p^n} - X = \prod_{d | n} \prod_{P \in \mathcal{P}(n,p)} P.$
\end{center}
\end{prop}
\begin{proof}
\begin{itemize}
\item Soit $P$ un facteur irréductible de $X^{p^n} - X$ sur $\F_p$ de degré $d$. Le corps de rupture de $P$ sur $\F_p$ est de cardinal $p^d$ du corps de décomposition $X^{p^n} - X$ sur $\F_p$, c'est-à-dire $F_{p^n}$, donc $d$ divise $n$.
\item Réciproquement, on suppose que $d$ divise $n$ et soit $P \in \mathcal{P} (n,p)$. Soit $\alpha$ une racine de $P$ dans le corps de rupture de $P$ sur $\F_p$. Alors on a $\F_p (\alpha) \simeq \F_{p^d}$. D'où $\alpha$ est racine de $X^{p^n} - X$. Or, puisque $P$ est irréductible, alors $P$ est le polynôme minimal de $\alpha$ sur $\F_p$ donc $P$ divise $X^{p^n} - X$. En outre les facteurs irréductible de $X^{p^n} - X$ sur $\F_p$ sont simples puisque $P$ est le polynôme minimal de $\alpha$ et que $P$ divise $X^{p^n} - X$.
\end{itemize}
\end{proof}
\begin{coro}
Soit $n \in \N^{*}$, il existe un polynôme irréductible de degré $n$ sur $\F_p$.
\end{coro}
\begin{proof}
En conservant les notations de la proposition précédente, il s'agit de montrer que $\# \mathcal{P} (n,p) > 0$. Pour ce faire on évalue le degré de l'égalité
\begin{center}
$\displaystyle X^{p^n} - X = \prod_{d | n} \prod_{P \in \mathcal{P}(n,p)} P.$
\end{center}
on a alors
\begin{center}
$\displaystyle p^n = \sum_{d | n} d \cdot \# \mathcal{P} (n,p)$
\end{center}
On en déduit alors que pour tout $d \in \N^{*}$ on a $p^d \geq d \cdot \# \mathcal{P} (n,p)$, puis,
\begin{align*}
\displaystyle
n \cdot \# \mathcal{P} (n,p) &= p^n - \sum_{d | n , d \neq n} d \cdot \# \mathcal{P} (n,p)\\
&\geq p^n - \sum_{d | n , d \neq n} p^d\\
&\geq p^n - \sum_{d =1}^{n-1} p^d\\
&\geq p^n - p \frac{p^{n-1} - 1}{p-1} > 0
\end{align*}
Puisque $n$ est positif alors $\mathcal{P} (n,p) > 0$.
\end{proof}




\subsection{Polynômes multivariés}


%Parler de LT, LM, ..., multideg ? ordre lexicographique

Dans ce qui suit, $\K$ désignera un corps quelconque.

\begin{definition}[Ordre]
Soit $E$ un ensemble quelconque, on appelle \textit{ordre partiel} sur $E$ toute relation vérifiant les propriétés suivantes pour $(x,y) \in E^2$:
	 \begin{enumerate}
	 	\item $x\ord x$ (réflexivité)
	 	\item $x\ord y$ et $y\ord x \Rightarrow x=y$ (antisymétrie)
	 	\item $x\ord y$ et $y\ord z \Rightarrow x\ord z$ (transitivité)
	 \end{enumerate}
En d'autres terme, $\ord$ est une relation d'équivalence sur $E$.
\end{definition}

\begin{definition}[Ordre total, ordonné]
Sous les mêmes notations, on dit que $\ord$ est un \textit{ordre total} si deux éléments quelconques sont toujours comparable, i.e. si
	\begin{center}$\forall (x,y)\in E^2, x\ord y\text{ ou } y\ord x$\end{center}
De plus, $\ord$ est dit \textit{bien ordoné} si
	\begin{center}$  \forall\text{ } F \subseteq E, \exists \text{ } f_{min}\in F \text{ tel que } \forall f\in F, \text{ } f_{min}\ord f $\end{center}
\end{definition}

\begin{definition}[Monoïde]
On appelle monoïde tout ensemble muni d'une loi de composition interne et d'un élément neutre.
\end{definition}

\begin{definition}
Soient $n\in\N$ et $\{X_1, ..., X_n\}$ un ensemble fini d'indéterminées. On définit le monoïde $\M_n$ comme suit:
	\begin{center} $\M_n := \{ X^{\alpha} := X_1^{\alpha_1}\ldots X_n^{\alpha_n}\}$ \end{center}
\end{definition}

\begin{prop}
\fonction{\phi}{\M_n}{\N^n}{ X^{\alpha} := X_1^{\alpha_1}\ldots X_n^{\alpha_n}}{\alpha:=(\alpha_1,\ldots, \alpha_n)}
est un isomorphisme de monoïde.
\end{prop}

\begin{definition}[Ordre monomial]
On dit que $\ord$ est un ordre monomial sur $\M_n$ si 
	\begin{enumerate}
		\item $\ord$ est un ordre total
		\item $\ord$ est compatible avec la multiplication, i.e. si pour tout
		\begin{center} $X=X_1\ldots X_n  \text{ , } \alpha=(\alpha_1,\ldots, \alpha_n) \text{ , } \beta=(\beta_1,\ldots, \beta_n) \text{ et } \gamma=(\gamma_1,\ldots, \gamma_n)$\end{center}
on a 
		\begin{center}$X^{\alpha} \ord X^{\beta }\Rightarrow X^{\alpha}.X^{\gamma} \ord X^{\beta}.X^{\gamma}$\end{center}
		\item $\M_n$ est bien ordonné par $\ord$
	\end{enumerate}
\end{definition}

\begin{definition}[Ordre lexicographique]
Pour deux vecteurs exposant $\alpha=(\alpha_1,\ldots, \alpha_n) \text{ et } \beta=(\beta_1,\ldots, \beta_n) \in\N^n$, on peut spécifier ordre, appellé \textit{ordre lexicographique} définit comme suit:
	\begin{center} $\alpha\ord_{lex} \beta $\end{center}
	\begin{center} si \end{center}
	\begin{center} $\exists \text{ } m\in\llbracket 1,n \rrbracket \text{ tel que } \forall \text{ } i<m \text{ , } \alpha_i-\beta_i=0  \text{ et } \alpha_m < \beta_m$ \end{center}
\end{definition}

Nous allons dès à présent travailler dans $\K[X_1, \ldots, X_n]$, et $\ord$ désignera toujours un ordre monomial sur $\M_n\subseteq \K[X_1, \ldots, X_n]$.

\begin{definition}[Leading Term]
On appelle \textit{terme} tout éléments de $\M_n$ multiplié par un élément non nul $c$ du corps de base. \newline
On appelle \textit{Leading Term}(LT) de $P\in \K[X_1, \ldots, X_n]$, son monôme de plus haut dégre par rapport à l'ordre $\ord$. \newline
La constante $c$ sera appellée \textit{Leading Coefficient}(LC), et $X_1^{\alpha_1}\ldots X_n^{\alpha_n}$ le \textit{Leading Monomial}, de sorte que :
\begin{center} $P = \underbrace{\underbrace{c}_\textrm{LC(P)}.\underbrace{X_1^{\alpha_1}\ldots X_n^{\alpha_n}}_\textrm{LM(P)}}_\textrm{LT(P)}$  + Q \end{center}
où $Q\in \K[X_1, \ldots, X_n]$ est constitué des termes de la forme $X^\beta \text{ , } \beta \ord_{lex} \alpha \text{ et } X \in\K[X_1, \ldots, X_n]$.
\end{definition}

\begin{definition}[Multi degré]
Le vecteur d'exposant $\alpha := (\alpha_1,\ldots, \alpha_n)$ est appellé le multi degré de $P$ et est noté $mdeg(P)$.
\end{definition}

\begin{prop}
Soient $P,Q \in\K[X_1, \ldots, X_n]$, on a $mdeg(P.Q) = mdeg(P) + mdeg(Q)$.
\end{prop}

\pagebreak

\section{Les polynômes symétriques}

\subsection{Introduction aux polynômes symétriques}
Les polynômes symétriques prennent forme à partir de l'étude des racines de n'importe quel polynôme. Considérons le polynôme $P=X^3 + bX^2 + cX + d$. C'est un polynôme cubique donc il a $3$ racines, non nécessairement distinctes. On notera ces racines $\alpha_1 , \alpha_2$ et $\alpha_3$.
Le polynôme $P$ peut alors se factoriser ainsi :
$$X^3 + bX^2 + cX + d = (X - \alpha_1)(X - \alpha_2)(X - \alpha_3),$$
ce qui nous donne :
\begin{eqnarray}
X^3 + bX^2 + cX + d &=& (X - \alpha_1)(X - \alpha_2)(X - \alpha_3) \nonumber \\
X^3 + bX^2 + cX + d &=& X^3 - X^2(\alpha_3 + \alpha_2 + \alpha_1) + X(\alpha_2 \alpha_3 + \alpha_1 \alpha_3 + \alpha_1 \alpha_2) - \alpha_1 \alpha_2 \alpha_3 \nonumber
\end{eqnarray}
Par identification, on obtient 
\begin{eqnarray}
b &=& -(\alpha_3 + \alpha_2 + \alpha_1) \nonumber \\
c &=& \alpha_2 \alpha_3 + \alpha_1 \alpha_3 + \alpha_1 \alpha_2 \nonumber \\
d &=& - \alpha_1 \alpha_2 \alpha_3. \nonumber 
\end{eqnarray}
On observe donc que les coefficients de $P$ sont polynomiaux en ses racines. Par ailleurs, comme modifier l'ordre des termes de $P$ ne le change pas, il s'ensuit que les polynômes définissant $b,c$ et $d$ par rapport à $\alpha_1$, $\alpha_2$ et $\alpha_3$ restent les mêmes si on permute $\alpha_1$, $\alpha_2$ et $\alpha_3$.\\
Les polynômes respectant ce fait sont dits \textit{polynômes symétriques}. Cela nous amène à la définition générale suivante.\\

%Peut-être réintroduire Sn l'ensemble des permutations à n éléments pour reprendre la definition de la première page de la ref 2

\begin{definition}
Un polynôme $P \in K\left[ X_1, X_2, \ldots , X_n \right] $ est dit symétrique si $$P(X_{i_1}, X_{i_2}, \ldots , X_{i_n}) = P(X_1,X_2,\ldots ,X_n),$$ pour toute permutation $X_{i_1}, X_{i_2}, \ldots , X_{i_n}$ de $X_1,X_2, \ldots ,X_n$.
\end{definition}
\vspace{12pt}
\begin{examples}
\begin{enumerate}
	\item Soit $P = X^{n_1} + Y^{n_2} + Z^{n_3}\in K\left[ X,Y,Z\right]$, avec $n_1,n_2,n_3 \in \mathbbm{N}$. Alors $P$ est un polynôme symétrique. En effet, 
		\begin{eqnarray}
P &=& X^{n_1} + Z^{n_3} + Y^{n_2} \nonumber \\
P &=& Y^{n_2} + X^{n_1} + Z^{n_3} \nonumber \\
P &=& Y^{n_2} + Z^{n_3} + X^{n_1} \nonumber \\
P &=& ... \nonumber
		\end{eqnarray}
		
	\item Soit $P = XYZ \in K\left[ X,Y,Z\right]$. Ce polynôme est symétrique car $P = XYZ = YZX = ZYX = \ldots$.
\end{enumerate}

\end{examples}

\subsection{Le théorème fondamental des polynômes symétriques}
En considérant tous les rappels faits à précédemment, nous pouvons introduire le fameux théorème fondamental des polynômes symétriques.

\begin{theorem}
Tout polynôme symétrique de $K\left[ X_1, X_2, ... , X_n \right]$ peut s'écrire de façon unique comme une expression polynomiale en les polynômes symétriques élémentaires $\sigma_1, \sigma_2,..., \sigma_n$.
\end{theorem}




\begin{proof}
Pour cette démonstration nous allons utiliser l'ordre lexicographique suivant, $x_1 > x_2 > \ldots > x_n$. Soit $f \in K[x_1, \ldots , x_n]$ un polynôme symétrique non nul, et on définit l'application $LT$ par $LT(f) = ax^{\alpha}$, où $\alpha=(\alpha_1 , \alpha_2, \ldots , \alpha_n )$ et $a \in K$. On peut supposer sans perte de généralité que les $\alpha_i$, $i\in \lbrace 1, \ldots , n \rbrace $ sont ordonnés comme tel : $\alpha_1 \geq \alpha_2 \geq \ldots \geq \alpha_n$.
En effet, supposons que l'on ait $\alpha_i < \alpha_{i+1}$ pour un certain $i\in \lbrace 1, \ldots , n \rbrace $. Il suffit alors de considérer le vecteur d'exposants $\beta$, obtenu à partir de $\alpha$ en permutant $\alpha_i$ et $\alpha_{i+1}$. On écrit $\beta = (\alpha_1 ,\ldots , \alpha_{i+1}, \alpha_i , \ldots , \alpha_n)$. Puisque $ax^{\alpha}$ est un terme de $f$, on en déduit que $ax^{\beta}$ est un terme de $f(x_1 , \ldots , x_{i+1}, x_{i}, \ldots , x_n)$. Or, $f$ est symétrique donc $f(x_1 , \ldots , x_{i+1}, x_{i}, \ldots , x_n)=f$, et par conséquent, $ax^{\beta}$ est un terme de $f$. Ceci est impossible puisque $\beta > \alpha$ selon l'ordre lexicographique.\\

Posons maintenant,
\begin{center}
$h=\sigma_1^{\alpha_1 - \alpha_2} \sigma_2^{\alpha_2 - \alpha_3} \ldots \sigma_{n-1}^{\alpha_{n-1}-\alpha_n} \sigma_n^{\alpha_n}$
\end{center}
Pour trouver le \textsc{Leading Term} de $h$, on a besoin de $LT(\sigma_r)=x_1x_2 \cdots x_r$ avec $r \in \lbrace 1, \ldots , r \rbrace$. On en déduit alors que, 
\begin{eqnarray}
LT(h) &=& LT( \sigma_1^{\alpha_1 - \alpha_2} \sigma_2^{\alpha_2 - \alpha_3} \ldots \sigma_{n-1}^{\alpha_{n-1}-\alpha_n} \sigma_n^{\alpha_n}) \nonumber \\
&=& LT(\sigma_1)^{\alpha_1 - \alpha_2}LT(\sigma_2)^{\alpha_2 - \alpha_3} \ldots LT(\sigma_n)^{\alpha_n} \nonumber \\
&=& x_1^{\alpha_1 - \alpha_2}(x_1x_2)^{\alpha_2 - \alpha_3} \ldots (x_1x_2 \ldots x_n)^{\alpha_n} \nonumber \\ 
&=& x_1^{\alpha_1}x_2^{\alpha_2} \ldots x_n^{\alpha_n} = x^\alpha . \nonumber
\end{eqnarray}
Il s'ensuit donc que $f$ et $ah$ ont le même \textsc{Leading Term}, et par conséquent,
\begin{center}
multideg($f-ah$) $<$ multideg($f$), lorsque $f-ah \neq 0$.
\end{center}
\vspace{12pt}Posons maintenant $f_1 = f - ah$. On remarque que $f_1$ est symétrique puisque $f$ et $ah$ le sont. Donc, si $f_1 \neq 0$, on peut répéter l'étape précédente pour construire $f_2 = f_1 - a_1h_1$, où $a_1$ est une constante et $h_1 = \displaystyle \prod_{i=1}^{n} \sigma_i^{\gamma_i}$, $\gamma_i \in \mathbbm{N}$. On sait aussi que $LT(f_2)<LT(f_1)$ lorsque $f_2 \neq 0$. En continuant ainsi on obtient une suite de polynômes $f, f_1, f_2, \ldots$ avec
\begin{center}
multideg($f$ )$>$ multideg($f_1$) $>$ multideg($f_2$) $\ldots$ .
\end{center}
Comme l'ordre lexicographique est bien ordonné, la suite est finie. Mais le processus se termine seulement lorsque $f_{t+1}=0$ pour un certain $t \in \mathbbm{N}$. On voit alors assez naturellement que 
$$f = ah + a_1h_1 + \ldots + a_th_t\,$$
ce qui montre que $f$ est polynomiale en les polynômes symétriques élémentaires.
\\

Il nous reste à montrer l'unicité. Supposons qu'on a un polynôme symétrique $f$ pouvant s'écrire
$$f=g_1(\sigma_1, \ldots , \sigma_n) = g_2(\sigma_1, \ldots , \sigma_n).$$
Notons $y_1, \ldots , y_n$ les $n$ variables des polynômes à $n$ indéterminées $g_1$ et $g_2$. On doit montrer que $g_1 = g_2$ dans $K \left[  y_1, \ldots , y_n \right] $.\\
Si on pose $g = g_1 - g_2$, alors $g(\sigma_1 , \ldots , \sigma_n) = 0$ dans $K \left[ x_1, \ldots , x_n \right] $. La preuve revient alors à montrer que $g=0$ dans $K \left[ x_1, \ldots , x_n \right]$.\\

Par l'absurde, supposons que $g \neq 0$. Si on écrit $g = \sum_{\beta} a_{\beta}y^{\beta}$, alors $g(\sigma_1, \ldots , \sigma_n)$ est la somme des polynômes $g_\beta = a_\beta \sigma_1^{\beta_1} \sigma_2^{\beta_2} \ldots \sigma_n^{\beta^n}$, où $\beta = (\beta_1 , \ldots , \beta_n)$. De plus, par le calcul de $LT(h)$, on déduit que $$LT(g_\beta) = a_\beta x_1^{\beta_1 + \ldots + \beta_n} x_2^{\beta_2 + \ldots + \beta_n} \ldots x_n^{\beta_n}.$$
Montrons maintenant que l'application,
\begin{center}
$
\begin{array}{l|rcl}
\iota :
    & (\beta_1 , \ldots , \beta_n) & \longmapsto & (\beta_1 + \ldots + \beta_n , \beta_2 + \ldots + \beta_n , \ldots , \beta_n) \end{array}
$
\end{center} est injective. Soient $\beta = (\beta_1 , \ldots , \beta_n)$ et $\beta ' = (\beta_1 ', \ldots , \beta_n ')$,
\begin{align*}
\iota (\beta) = \iota (\beta ') &\Leftrightarrow (\beta_1 + \ldots + \beta_n , \beta_2 + \ldots + \beta_n , \ldots , \beta_n) = (\beta_1 ' + \ldots + \beta_n ' , \beta_2 ' + \ldots + \beta_n ' , \ldots , \beta_n ')\\
&\Leftrightarrow \left \{
\begin{array}{rcl}
\beta_1 + \ldots + \beta_n  &=& \beta_1 ' + \ldots + \beta_n ' \\
\beta_2 + \ldots + \beta_n  &=& \beta_2 ' + \ldots + \beta_n ' \\
&...&\\
\beta_n &=& \beta_n '
\end{array}
\right.\\
&\Leftrightarrow \beta_i = \beta_i' \hspace{0,2cm} , \hspace{0,2cm} \forall i \in \{ 1,  \ldots , n \} , \hspace{0,2cm} \text{en remontant les égalités de chaque ligne}
\end{align*}

Donc $\iota$ est une application injective. Par conséquent, les $g_\beta$ ont des \textsc{Leading Term} distincts. En particulier, en choisissant $\beta$ tel que $LT(g_\beta)>LT(g_\gamma)$, quelques soient $\gamma \neq \beta$, alors $LT(g_\beta)$ sera plus grand que tous les termes des $g_\gamma$. Finalement il n'y a rien pour annuler $LT(g_\beta)$, et par conséquent, $g(\sigma_1 , \ldots , \sigma_n)$ ne peut être nul, l'unicité en découle.
\end{proof}


\begin{corollaire}

\end{corollaire}





\pagebreak
\addcontentsline{toc}{part}{Références}

\begin{thebibliography}{9}
	\bibitem{these1}
	COX David, LITTLE John, O'SHEA Donal\\
	Ideal, Varieties, and Algorithms - An Introduction To Computational Algebraic Geometry and Commutative Algebra,\\
	Third Edition,
	7.1, p.317- ?

	\bibitem{internet}
	\url{https://math.unice.fr/~walter/L3_Alg_Arith/cours2.pdf}	
\end{thebibliography}

\end{document}