\documentclass[12pt]{article}
%\usepackage{natbib}
\usepackage[french]{babel}
\usepackage{url}
\usepackage[utf8x]{inputenc}
\usepackage{graphicx}
\graphicspath{{images/}}
\usepackage{parskip}
\usepackage{fancyhdr}
\usepackage{vmargin}
\usepackage{xcolor}
\usepackage{bbm}
\usepackage{amsmath,amsthm,amssymb,latexsym,amsfonts}
\usepackage{dsfont}
\usepackage{stmaryrd}
\usepackage{systeme}
\usepackage{enumitem}
\usepackage{xcolor}
\usepackage{pifont}
%\usepackage[cache=false]{minted}
%\definecolor{LightGray}{gray}{0.95}
\usepackage{autobreak}

\title{Théorème fondamental des polynômes symétriques}
\author{PIARD A. - JACQUET R. - CARVAILLO T.}
\date{\today}

\makeatletter
\let\thetitle\@title
\let\theauthor\@author
\let\thedate\@date
\makeatother

\pagestyle{fancy}
\fancyhf{}
\rhead{\theauthor}
\lhead{\thetitle}
\cfoot{\thepage}
\def\dotfill#1{\cleaders\hbox to #1{.}\hfill}
\newcommand\dotline[2][.5em]{\leavevmode\hbox to #2{\dotfill{#1}\hfil}}

%définition commande présentation fonction
\newcommand{\fonction}[5]{
\begin{displaymath}
\begin{array}{l|rcl}
\displaystyle
#1 : & #2 & \longrightarrow & #3 \\
    & #4 & \longmapsto & #5
\end{array}
\end{displaymath}
}
%fin définition
\theoremstyle{remark}\newtheorem{note}{Note}
\theoremstyle{remark}\newtheorem{nota}{Notation}

\newcommand{\M}{\mathbbm{M}}
\newcommand{\N}{\mathbbm{N}}
\newcommand{\Z}{\mathbbm{Z}}
\newcommand{\Q}{\mathbbm{Q}}
\newcommand{\R}{\mathbbm{R}}
\newcommand{\C}{\mathbbm{C}}
\newcommand{\G}{\mathbbm{G}}
\newcommand{\K}{\mathbbm{K}}
\newcommand{\F}{\mathbbm{F}}
\newcommand{\Fq}{\mathbbm{F}_q}
\newcommand{\Fqn}{\mathbbm{F}_{q^n}}
\newcommand{\Fp}{\mathbbm{F}_p}

%fin définition


% de jolies accolades
\newcommand{\accolade}[2]{
\begin{displaymath}
%#1 = \left\{
    \begin{array}{ll}
       #1 %& \mbox{si }
       #2 %& \mbox{sinon.}
    \end{array}
\right.
\end{displaymath}
}
% de jolies accolades


\newtheorem{theorem}{Théorème}
\newtheorem{corollaire}{Corollaire}
\newtheorem{lemma}{Lemme}
\newtheorem{prop}{Proposition}
\theoremstyle{definition}
\newtheorem{definition}{Définition}
\newtheorem{example}{Exemple}
\newtheorem*{examples}{Exemples}
\newtheorem{exo}{Exercice}	
\newtheorem{coro}{Corollaire}	
\newtheorem{rem}{Remarque}
\newtheorem{crit}{Critère}
\newtheorem{bg}{A l'attention des bg, question}

\begin{document}
%%%%%%%%%%%%%%%%%%%%%%%%%%%%%%%%%%%%%%%%%%%%%%%%%%%%%%%%%%%%%%%%%%%%%%%%%%%%%%%%%%%%%%%%%

\begin{titlepage}
	\centering
    \vspace*{0.5 cm}
    \textsc{\LARGE Projet de Systèmes Polynomiaux.\\
    \vspace{12pt}
2020-2021}\\[1.0 cm]
    \dotline[15pt]{15cm}\\
	\includegraphics[scale = 2.2]{logo.png}
	\dotline[15pt]{15cm}\\
	\vspace{1.5cm}
	\textsc{\Large Faculté des Sciences et Techniques}\\
	\textsc{\large Master 1 - Maths. CRYPTIS}\\[1.0 cm]
	\rule{\linewidth}{0.2 mm} \\[0.4 cm]
	{ \huge \bfseries \color{blue} \thetitle}\\
	\rule{\linewidth}{0.2 mm} \\[1.5 cm]
	
	\begin{minipage}{0.4\textwidth}
		\begin{flushleft} \large
			\emph{A l'attention de :}\\
			M. LICKTEIG\\
			\phantom{a}\\
			\phantom{a}\\
		\end{flushleft}
	\end{minipage}
	\begin{minipage}{0.5\textwidth}
    	\begin{flushright} \large
		\emph{Rédigé par :}\\
		PIARD A.\\
		JACQUET R.\\
		CARVAILLO T.\\
		\end{flushright}
	\end{minipage}\\[2 cm]
\end{titlepage}

%%%%%%%%%%%%%%%%%%%%%%%%%%%%%%%%%%%%%%%%%%%%%%%%%%%%%%%%%%%%%%%%%%%%%%%%%%%%%%%%%%%%%%%%%

\tableofcontents
\pagebreak


\section{Rappels sur les Corps Finis} %et polynômes en plusieurs variables ?






% Rappels sur les corps finis




%Parler de LT, LM, ..., multideg ?




Dans la suite de ce rapport, $K$ désignera un %corps ?^
\pagebreak

\section{Les polynômes symétriques}

\subsection{Introduction aux polynômes symétriques}
Les polynômes symétriques prennent forme à partir de l'étude des racines de n'importe quel polynôme. Considérons le polynôme $P=X^3 + bX^2 + cX + d$. C'est un polynôme cubique donc il a $3$ racines, non nécessairement distinctes. On notera ces racines $\alpha_1 , \alpha_2$ et $\alpha_3$.
Le polynôme $P$ peut alors se factoriser ainsi :
$$X^3 + bX^2 + cX + d = (X - \alpha_1)(X - \alpha_2)(X - \alpha_3),$$
ce qui nous donne :
\begin{eqnarray}
X^3 + bX^2 + cX + d &=& (X - \alpha_1)(X - \alpha_2)(X - \alpha_3) \nonumber \\
X^3 + bX^2 + cX + d &=& X^3 - X^2(\alpha_3 + \alpha_2 + \alpha_1) + X(\alpha_2 \alpha_3 + \alpha_1 \alpha_3 + \alpha_1 \alpha_2) - \alpha_1 \alpha_2 \alpha_3 \nonumber
\end{eqnarray}
Par identification, on obtient 
\begin{eqnarray}
b &=& -(\alpha_3 + \alpha_2 + \alpha_1) \nonumber \\
c &=& \alpha_2 \alpha_3 + \alpha_1 \alpha_3 + \alpha_1 \alpha_2 \nonumber \\
d &=& - \alpha_1 \alpha_2 \alpha_3. \nonumber 
\end{eqnarray}
On observe donc que les coefficients de $P$ sont polynomiaux en ses racines. Par ailleurs, comme modifier l'ordre des termes de $P$ ne le change pas, il s'ensuit que les polynômes définissant $b,c$ et $d$ par rapport à $\alpha_1$, $\alpha_2$ et $\alpha_3$ restent les mêmes si on permute $\alpha_1$, $\alpha_2$ et $\alpha_3$.\\
Les polynômes respectant ce fait sont dits \textit{polynômes symétriques}. Cela nous amène à la définition générale suivante.\\

%Peut-être réintroduire Sn l'ensemble des permutations à n éléments pour reprendre la definition de la première page de la ref 2

\begin{definition}
Un polynôme $P \in K\left[ X_1, X_2, \ldots , X_n \right] $ est dit symétrique si $$P(X_{i_1}, X_{i_2}, \ldots , X_{i_n}) = P(X_1,X_2,\ldots ,X_n),$$ pour toute permutation $X_{i_1}, X_{i_2}, \ldots , X_{i_n}$ de $X_1,X_2, \ldots ,X_n$.
\end{definition}
\vspace{12pt}
\begin{examples}
\begin{enumerate}
	\item Soit $P = X^{n_1} + Y^{n_2} + Z^{n_3}\in K\left[ X,Y,Z\right]$, avec $n_1,n_2,n_3 \in \mathbbm{N}$. Alors $P$ est un polynôme symétrique. En effet, 
		\begin{eqnarray}
P &=& X^{n_1} + Z^{n_3} + Y^{n_2} \nonumber \\
P &=& Y^{n_2} + X^{n_1} + Z^{n_3} \nonumber \\
P &=& Y^{n_2} + Z^{n_3} + X^{n_1} \nonumber \\
P &=& ... \nonumber
		\end{eqnarray}
		
	\item Soit $P = XYZ \in K\left[ X,Y,Z\right]$. Ce polynôme est symétrique car $P = XYZ = YZX = ZYX = \ldots$.
\end{enumerate}

\end{examples}

\subsection{Le théorème fondamental des polynômes symétriques}
En considérant tous les rappels faits à précédemment, nous pouvons introduire le fameux théorème fondamental des polynômes symétriques.

\begin{theorem}
Tout polynôme symétrique de $K\left[ X_1, X_2, ... , X_n \right]$ peut s'écrire de façon unique comme une expression polynomiale en les polynômes symétriques élémentaires $\sigma_1, \sigma_2,..., \sigma_n$.
\end{theorem}







\pagebreak
\addcontentsline{toc}{part}{Références}

\begin{thebibliography}{9}
	\bibitem{these1}
	COX David, LITTLE John, O'SHEA Donal\\
	Ideal, Varieties, and Algorithms - An Introduction To Computational Algebraic Geometry and Commutative Algebra,\\
	Third Edition,
	7.1, p.317- ?

	\bibitem{internet}
	\url{https://math.unice.fr/~walter/L3_Alg_Arith/cours2.pdf}
	
\end{thebibliography}
\end{document}